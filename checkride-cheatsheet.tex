%% Checkride cheat-sheet
%%
%% Author:
%%   Filip Miletic (filmil@gmail.com)
%%
%% WARNING: 
%%   This was made for my personal use.  Not to be used in lieu of 
%%   actual aircraft procedures from the approved flight manual or
%%   pilot's operating handbook.

\documentclass[article,9pt,landscape]{memoir}
\usepackage{multicol,lipsum}
\usepackage[footnote]{acronym}
\usepackage[ %% Letter paper, uses almost the entire paper surface
  text={10.9in,8.3in},
  centering,ignoreall]{geometry}


\begin{document}
% -*- latex -*-

%%%%%%%%%%%%%%%%%%%%%%%%%%%%%%%%%%%%%%%%%%%%%%%%%%%%%%%%%%%%
%% Commands

% One procedure item
\newcommand{\procitem}[2]{{#1}\dotfill{\textsc{#2}}\newline} 
\newcommand{\plane}[1]{{\textbf{\textsf{#1}}}\newline} 
\newcommand{\readback}{\plane{Readback}}
\newcommand{\morse}[1]{$|$\texttt{{#1}}}


%%%%%%%%%%%%%%%%%%%%%%%%%%%%%%%%%%%%%%%%%%%%%%%%%%%%%%%%%%%%
%% Page layout
\setlength{\parindent}{0em}
\pagestyle{empty}
\makechapterstyle{cheatsheet}{%
  \renewcommand*{\chaptitlefont}{\bfseries\Large\sffamily}
  \renewcommand*{\chapterheadstart}{}
  \renewcommand*{\printchaptername}{}
  \renewcommand*{\chapternamenum}{}
  \renewcommand*{\printchapternum}{}
  \renewcommand*{\afterchapternum}{}
  \renewcommand*{\beforechapskip}{0.5em}
  \renewcommand*{\afterchapskip}{0.5em}
}
\chapterstyle{cheatsheet}

%% Acronym list

\begin{acronym}
  \acro{mca}[MCA]{Minimum Controllable Airspeed}
\end{acronym}



%%% Document titling

\begin{multicols*}{4}
  \textsc{Checkride Cheat Sheet C-172SP\\\today\\[0.5em]}

  \chapter{At begin taxi}
  Always check brake at taxi begin.

  \chapter{Before maneuver}
  \plane{Clearing turn} 
  Clearing turn either 2 times 90 degrees, or 1
  time 180 degrees.  Always before any maneuver.

  \chapter{Divert}
  \plane{Setup}
  While determining divert parameters, keep altitude; keep circling.  Choose
  visual reference to start from and stay over it.

  If the general direction is known, select visual reference in that
  general direction.
  
  \plane{Autopilot}
  \textsc{AP} to turn on autopilot.  Then \textsc{ALT} to keep
  altitude; Then \textsc{HDG} to link heading bug of the heading
  indicator to the autopilot.
  
  \begin{enumerate}
    \item \textbf{Reference}: choose prominent visual reference, far
      away from airspace boundaries

    \item \textbf{Legs to fly}: check hills/terrain to determine legs;
      anchor legs to prominent visual references; then for each leg

    \item \textbf{Heading} to fly required

    \item \textbf{Ground speed}: correct for winds aloft at altitude

    \item \textbf{Time} required

    \item \textbf{Fuel} required

    \item \textbf{Airspace} ceilings/limitations

    \item \textbf{Comm frequencies}: ATIS/ASOS/AWOS destination
      airport; CTAF or tower frequency; enroute airports;
      approach/departure controls.

    \item \textbf{Pattern altitude} of destination airport

    \item \textbf{Setup NAV}: if helpful NAVs are available, set them up.

    \item \textbf{Setup GPS}: if GPS is allowed, set up GPS.
  \end{enumerate}
  \chapter{Landing}
  \chapter{Class G entry}
  Enter pattern at pattern altitude.  Enter at the 45 leg.  Honor left traffic
  where appropriate, right traffic where appropriate.

  \chapter{Instrument check}

  \begin{itemize}
    \item Airspeed zero

    \item turn coordinator wing down ball to the outside (in a turn)

    \item attitude ind less than 5 degrees tilt

    \item heading indicator swings freely

    \item altimeter within 75 feet

    \item VSI zero, 

    \item magnetic compass no leaks, swings freely, there's a
      deviation card and shows known headings.
  \end{itemize}

  \chapter{Passenger briefing (before engine start)}

  \begin{itemize}
    \item If you feel sick let me know right away.

    \item Do not touch any controls including the foot pedals.

    \item No smoking.

    \item Demonstrate how to open and close the door and point out all
      emergency exits (including baggage door).

    \item Demonstrate how to use the seatbelts and shoulder harnesses
      and say they must be on at all times.

    \item If you hear bells or horns from the plane don't be
      alarmed. It's part of the flight.

    \item If you would like you can participate by looking for other
      airplanes in the sky and pointing them out.

    \item When I raise my hand no talking. If you hear my callsign on
      the radio do not talk.

    \item No talking below 1000 feet, especially if I am taking off or
      coming in to land.
  \end{itemize}

  \chapter{Take off/Pilot Briefing}

  \begin{itemize}
    \item If we have an engine failure while on the runway, I will
      throttle back and stop.
  
    \item If we have an engine failure after rotation I will land on
      the remaining runway.

    \item If there is no more usable runway I will pitch down and land
      straight ahead.
  \end{itemize}

  %% -*- latex -*-

  \vspace{\stretch{1}}
  \plane{Disclaimer}  
  This sheet is made for personal use and is still work in
  progress.  It has not been checked for errors, it has not been
  tested in practice, it is not authoritative, and you should
  basically not use it.  

  \textbf{\textsf{Under no conditions should you use it as a primary
      authority on procedures and safety.  Your life and the lives of
      your passengers are your own responsibility, so act accordingly
      and refer to appropriate FAA-approved publications.}}


\end{multicols*}
\end{document}
